\section*{\zihao{3} \centering 摘要}

\vskip0.5cm
本设计旨在用最少的设备去模拟一个校园网的网络环境.在连接功能上能够实现内部所有终端能够访问外网,服务器对内网和外网均开放.安全方面实现不同部门之间隔离但是内部互通, 教学区和非教学区隔离.ip分配上实现服务器用静态ip分配, 其他终端用动态ip分配.服务器内容方面有校园网站,文件共享,内网dns三个服务.

\textbf{关键词:}  校园网 ACL,VLAN,DHCP,NAY, DNS
\addcontentsline{toc}{section}{摘要}

\clearpage
\section*{\zihao{2} \centering \textbf{Abstract} }
   %用了Times New Roman字体来美化观感
Our design aimed at simulating the network environment of a campus network with a minimum number of equipments.
On the point of connection function,all of our inside hosts can visit the Internet.What'more,the server is open to
both the intranet and the outer network.In terms of security ,what we have already relized is the isolation between
different departments but interoperability,isolation of teaching and non teaching areas.AS  for IP allocation,
the server is allocated in a static IP, and the other terminals are allocated with dynamic IP.Also,there are three services:campus web site,file sharing and internal network DNS.  

\textbf{Key Words:}Campus network ,ACL ,VLAN ,DHCP ,NAT ,DNS 
\addcontentsline{toc}{section}{Abstract}




