\pagenumbering{arabic}

\subsection{介绍}
\indent 大家好,我将介绍一种新的实现OFDM仿真的模型,并且和之前的模型进行对比,以验证二者之间的等价性.

\subsection{整体流程}
\indent ofdm的整理流程是,频域到时域再到频域.
\begin{itemize}
  \item 在频域中,原始序列经过调制得到频域数据,经过fft进入时域
  \item 在时域中,数据会历经多径衰落,高斯白噪声.当然为了消除码间干扰和符号间干扰,需要增减循环前缀.最后通过fft进入频域
  \item	在频域中,通过均衡和解调得到最终的输出序列
\end{itemize}

\subsection{数学实现}
\indent 既然是从频域到时域再到频域,那么能否将时域部分的所有操作封装起来,以实现输入频域信号X,经过一个系统直接得到频域的输出?
其关键就是把时域中的所有操作用矩阵来表示.
这样从数学的角度来看,就是如下三个公式.

原始数据乘以F逆矩阵再乘以信道矩阵,其频域变换加上噪声的频域变换就是最终结果.
当然最后还要进行信道均衡,均衡是通过预置在信号中的导言来实现的.

其中F,F逆和W比较好处理,比较棘手的是H,这也是我今天所讲的重点,我将介绍两种实现H的方式,并且验证二者之间的等价性.

\subsection{H的实现}
\subsubsection{方法一}
\indent 第一种,X为原始序列加上一部分循环体,其中循环体包含了循环前缀和循环延时.

举个栗子,当我求加上循环前缀的信号的第一个输出的时候,结果应当为$X(-N_{CP})$乘以延时为0的信道, 加上前一个数据,也就是$X(-N_{CP}-1)$乘以延时为1的信道, ...,加上最后一个数据乘以最大延时信道.
把这些系数合起来,就是H矩阵的第一行

当我在求第二个输出的时候,结果应当为$X(-N_{CP}+1)$乘以延时为0的信道,加上前一个数据,也就是$X(-N_{CP})$乘以延时为1的信道, ..., 加上倒数第二个数据乘以最大延时.
把这些系数合起来,就是H矩阵的第二行.


以此类推, 用这个行向量组成的循环矩阵就是H, H乘以X就是加上循环前缀的信号的输出.因为输出$N+N_{CP}$个符号, 所以H矩阵有$N+N_{CP}$行.
这也是我们之前作业中的方法.只不过我用矩阵来实现

\subsubsection{方法二}
\indent 可以发现在X中有冗余的部分,如果我们将这部分冗余通过修改H来去掉,就得到了第二种方法.

举个栗子, 当我求信号的第一个输出的时候, 结果应当为$X(0)$乘以延时为0的信道, 前一个数据,也就是$X(N-1)$乘以延时为1的信道, ...,一直到最大时延对应的那个数据.

当我在求第二个输出的时候, 结果应当为$X(1)$乘以延时为0的信道, 加上前一个数据, 也就是$X(0)$乘以延时为1的信道, ..., 一直到最大时延对应的那个数据.
把这些系数合起来, 就是H矩阵的第二行.

以此类推, 用这个行向量组成的循环矩阵就是H, H乘以X就是信号的输出.因为一共输出N个符号, 所以H矩阵有N行

可以发现, 在第二种方法中不会涉及到循环前缀, 这符合我们的预期, 因为循环前缀本身就是加了再去, 对信息本身没有影响.

\subsection{代码实现}
\indent 这是从理论方面验证两种方式的等价性, 接下来通过代码来检验.

检验的方式就是给定相同的X,相同的W,不同的H. 如果两个H具有等价性,那么输出的Y应当相同, 输出结果应当为:两种方式的得到的y\_bits差异值为0.

在看结果之前,先看看具体的实现方式

先看方法二,在配置好F,F逆,X,W矩阵后,直接一个式子得到结果, 一个式子用来均衡, 得到最终输出结果.

再来看方法一,相比于方法二, 该方法不会显得那么简洁, 因为要手动的增减循环前缀.

\subsection{最终结果}
这是最终的结果, 可以看到,两种方式的得到的y\_bits差异值为0, 可以证明二者的等价性.

另外相同的结果, 方法二却能够更优雅的实现而且更省资源.

这种封装有什么用呢?
历史上每一次的统一都能够给人们带来一个崭新的世界.
\begin{itemize}
  \item 牛顿力学定律让我们知道,天上飞的和地上跑得都遵循同一个规律.
  \item 傅立叶变换让我们知道看起来截然不同的信号可能只是在频域内的分量略有差别而已.
  \item 而这种封装,让我们知道,ofdm,cdma,其实也是之数学公式框架中的不同参数而已
\end{itemize}
